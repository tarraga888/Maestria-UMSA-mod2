\documentclass[Spanish,12pt,doublespace,german,letterpaper,dvipdfm]{article}

\usepackage[T1]{fontenc}
\usepackage[spanish]{babel}
\usepackage{latexsym}
\usepackage{enumerate}
\usepackage{amsmath}
\usepackage{amsbsy}
\usepackage{amssymb}
\usepackage{graphics}
\usepackage{graphicx}
\usepackage{color}
\usepackage{hyperref}
\DeclareGraphicsExtensions{.jpg} {
\renewcommand{\baselinestretch}{1.5}  %Espaciamiento de lineas
\frenchspacing %control de espacios

% Título Portada
\centerline{\Large UNIVERSIDAD MAYOR DE SAN ANDRÉS}
\centerline{FACULTAD DE CIENCIAS PURAS Y NATURALES}
\centerline{POSTGRADO AUTOFINANCIADO
EN MATEMÁTICA} 
\centerline{Módulo II: Matrices} 
\vspace{12em}
\centerline{\huge RAUIZ DE  MATRICES  MEDIANTE }\vspace{1em}
\centerline{\huge SCHUR APLICADO A VISION ARTIFICAL}\vspace{1em}
 \centerline{\huge  REJILLAS DE FRECUENCIA}\vspace{1em}
 \centerline{\huge (EN T.F. FOURIER ) }
\vspace{10em} \centerline{\Large EDGAR DANIEL TÁRRAGA
TORREZ}\vspace{4em}
 \centerline{\Large Tarija, julio 2023}\vspace{5em}

 \pagebreak

 \begin{document}
 \tableofcontents
 \pagenumbering{arabic}
%\listoftables
%\listoffigures
% quitar los `%' de la siguiente secci'on si quieres
% incluir un resumen de tu tesis.


\include{intro}

En el  procesamiento de imagenes la importancia de la Transformada de Fourier  se desarrollo con  métodos numéricos que permiten su uso en contexto computacional. 
La TF discreta permite encontrar una función espectral discreta (es decir, para una secuencia finita de N frecuencias) a partir de una señal discreta de N valores:


La misma formulación de la transformada discreta se aplica a señales 2D (imágenes), aprovechando otra propiedad fundamental de la TF, denominada separabilidad (la TF de una función 2D que puede ponerse como producto de dos funciones 1D, es el producto de cada una de las TF de dichas funciones).
De esa forma surge la TF discreta para funciones 2D f(nx, ny), donde Fh y Fv representan frecuencias horizontales y verticales, respectivamente (en este caso se trata de frecuencias geométricas):

 

Una forma de representar la TF de una imagen es con otra imagen de igual resolución. 
En si Fh y Fv representa una matriz nueva del espectro de frecuencias el cual se  utiliza  alguna forma de pseudocolor para representar la energía a cada una de las frecuencias el cual estas nos dan los paramteros de ruidos,nitidez,etc sobre la imagen.
Dicho esto el tratamiento de esta repesentacion grafica de forurier  tiene un campo grande donde existen numerosas propuestas aplicadas para mejorar imagenes. 

\subsection{Planteamiento del problema}
La TF de las imagenes con una interferencia aditiva se puede observae algunos  puntos simétricos donde esa interferencia está localizada en el espectro. 
Si editamos el espectro y eliminamos la energía a esas frecuencias se puede restaurar la imagen original.
En resumen una de las técnicas iníciales es hacer la captura de la imagen de la T.F. y determinar un aplicativo  manual que elimine los puntos blancos concentrados de energia y esto dara un efecto, modificatorio a  la TF y se realizia  la transformada inversa con el espectro manipulado( Imagen corregida ), y se logra corregir casi por completo el defecto en la imagen original.

\subsection{Justificaci\'on}
Mediante aplicativos de Python se pretende   encontrar  la matriz de modulo y angulo de la descomposion de Fourier en 2D , el cual con estas matrices se buscara en filas y columnas los puntos altos de concentracion de frecuencias y asi recuadrar rejillas 2x2 para sus tratamiento  
El tratamiento que se pretende usar las tecnicas de descomposiion de matrices  mediante schur (factorizacion)  estudiadas en el modulo II de matrices y asi mismo calcular la raiz matricial de esta rejilla y reemplazar en la matriz principal, este proceso de tratamiento de rejillas de 2x2 se hara con dos iteracciones.
\subsection{Objetivo General.}
El presente proyecto  tiene por objetivo desarrollar un aplicativo/formulacion  en python para hallar la raiz cuadrada  de una matrizy mediante schur ,asi mismo este tratamiento de raiz cuadrada se aplicara  a rejillas 2x2 en puntos maximos de la mtariz principal de frecuencias  el cual reemplazara en las rejillas/matrices 2x2, y con  este proceso de tratamiento de rejillas/matrices de 2x2 llegar a  dos iteracciones. para ver su mejoramiento

\subsection{Objetivos Especificos.}
Entre los objetivos especificos que se pretende alcanzar son:
\begin{itemize}
  \item Desarrollo de ecuaciones para calcular autovalores y autovectores en python para matrices 2x2 
  \item Desarrollo de ecuaciones para la descomposion de Grand Schmit.
  \item Desarrollo de de ecuaciones   para el calculo de raiz cuadra mediante schur
  \item Desarrollo de aplicativos de jupiter notebook para codificacion en  python de las ecuaciones a usar en este proyecto
  \item Desarrollo de aplicativos en python para visualizar el espectro de Foruier.
 \end{itemize}
 
\section{Fundamentos Teoricos}
\subsection{Polinomio caracteristiv}
Sea f : V → V una aplicaci ́on lineal con matriz asociada A en una cierta base.


Se sabe que 
$$f(x) = Ax$$ 
Si x es un autovector asociado a un autovalor λ, entonces
$$f(x) = λx ⇒ Ax = λx ⇒ (A − λI)x = 0$$
Denotando por $A = (a_{ij})$ y $x = (x_1,...,x_n)$, podemos escribir estas ecuaciones expl ́ıcitamente como
\begin{equation}
\begin{matrix}
 (a_{11}-\lambda)x_1    & + & a_{12}x_2          & +\cdots & a_{1n}x_n       & =  & 0\\
  a_{21}x_1             & + & (a_{22}-\lambda)x_2  & +\cdots & a_{2n}x_n       & =  & 0\\
 \vdots & \vdots        &        & \vdots & \vdots \\
  a_{n1}x_1             & + & a_{n2}x_2   & \cdots & (a_{nn}-\lambda)x_n &  =  & 0
\end{matrix}
\end{equation}
Obteniendo un sistema homog ́eneo que debe satisfacer cada uno de los auto- vectores asociados al autovalor λ.

As ́ı pues, para que existan autovectores, este sistema debe tener solucio ́n distinta de la trivial, o equivalentemente, la matriz del sistema debe tener rango menor que n. 
Esto es,

\begin{equation}
\begin{vmatrix}
  a_{11}-\lambda    & + & a_{12}          & +\cdots & a_{1n}       \\
  a_{21}            & + & a_{22}-\lambda  & +\cdots & a_{2n}       \\
 \vdots & \vdots        &        & \vdots & \vdots \\
  a_{n1}             & + & a_{n2}x_2   & \cdots & a_{nn}-\lambda 
\end{vmatrix}
\end{equation}

$$|A -\lambda I_n|=0$$
\subsection{El sistema de control de procesos en temperatura}
Considerándose un horno el\'ectrico ( mufla de laboratorio ) en el
cual la temperatura interna del proceso se calienta mediante
hornillas  internas como se muestra en la figura ,la cual dichas
hornillas son alimentadas por una
fuente de energia .\\

\begin{figure}[h!]
\begin{center}
 \includegraphics[width=10cm]{plani1_1.jpg}
\end{center}
\caption{ Horno Electrico }
\end{figure}

El propósito de la unidad es calentar el fluido(aire) del medio
que se procesa, de una temperatura de entrada $T_i(t)$, a cierta
temperatura $T(t)$ que se desea.\\
Como se dijo anteriormente el medio de calentamiento son las
hornillas el\'ectricas y la energía que gana el fluido el proceso
es igual al calor que libera las hornillas, siempre y cuando no
haya perdidas de calor en el entorno es decir el aislamiento
perfecto que
existe dentro del horno.\\
En este proceso existen muchas variables que pueden cambiar , lo
cual ocasiona que la temperatura de salida se desvié del valor
deseado ,si esto llega a suceder se deben de emprender algunas
acciones inmediatas para corregir la desviaci\'on ,esto es el
objetivo de controlar la temperatura de salida del proceso  para
mantener el
valor deseado.\\
Una manera de lograr este objetivo es primero medir la temperatura
$T(t)$, después comparar este con el valor que se desea  y con la
base en la  comparaci\'on decidir que se debe hacer para corregir
cualquier desviación .\\
Se puede usar el flujo del aire interno del horno para corregir la
desviación , es decir  si la temperatura esta por arriba del valor
deseado entonces se puede abrir el interruptor  para cortar el
flujo de energía de las hornillas .Si la temperatura esta por
debajo del valor que se desea se puede cerrar el interruptor para
aumentar el flujo de calor interno.Todo esto lo puede hacer
manualmente el operador el proceso.\\
Sin embargo la operación manual no dará una regulación perfecta
por ello seria preferible  realizar el control de manera
automática , es decir contar con instrumentos que controlen la
variables sin necesidad que intervenga el operador.\\Esto es lo
que
significa el control automático de procesos.\\

\subparagraph{Sistema de control propuesto} Para lograr el
objetivo de procesos automáticos se debe dise~nar
e implementar un sistema de control .\\
En la figura  se muestra el sistema de control propuesto y sus
componentes básicos.\\
El primer paso es medir la temperatura interna del horno
eléctrico, esto se hace mediante un sun sensor (PT100
,dispositivo de resistencia ).\\
El sensor se conecta físicamente al transductor, el cual capta  la
salida del sensor y la convierte en un se~nal lo suficiente
compatible para transmitir al controlador.\\
El controlador recibe la se~nal que esta en relación con la
temperatura , la compara con el valor que se desea y seg\'un el
valor de comparac\'ion , decide que hacer para mantener la
temperatura en el valor deseado .\\
Con base en la decision, el controlador envía otra se~nal al
elemento final de control (interruptor electrónico),el cual a su
vez maneja el flujo de energía del horno eléctrico.\\

\begin{figure}[h!]
\begin{center}
 \includegraphics[width=12cm]{plani1_3.jpg}
\end{center}
\caption{Sistema de control Horno Eléctrico }
\end{figure}

Los componentes b\'asicos del sistema de control descrito seria:
\begin{itemize}
  \item Sensor , que también se conoce como elemento primario
  \item Controlador, que es el cerebro del sistema de control
  \item Elemento final de control o actuador principal.
\end{itemize}
La importancia de estos componentes esta en que realizan las tres
operaciones básicas que deben estar presentes en todo sistema de
control , esta operaciones son:
\begin{description}
  \item[Medici\'on]  de la variable que se controla ,
  generalmente se la hace mediante la combinaci\'on de sensor y
  transmisor
  \item[Decisi\'on] con base a la medici\'on , el controlador
  decide que hacer para mantener la variable en el valor que se
  desea.
  \item[Acci\'on] como resultado de la decisi\'on del controlador
  se debe efectuar una regulaci\'on del elemento final.
\end{description}


\subsubsection{Control por realimentaci\'on} El concepto de control
por realimentación a pese de existir hace mas de mil a~nos no tuvo
aplicaci\'on practica en la industria hasta James Watt lo aplico
en el control de velocidad de su maquina de vapor,actualmente la
gran mayoría de los sistemas de control automático se incluye al
menos un circuito de control por retroalimentación.

\paragraph{Descripción del control por realimentaci\'on en el horno
eléctrico} El concepto de  control por realimentación en procesos
automáticos es como se describe a continuaci\'on :\\
El objetivo es mantener la temperatura del fluido (aire) que se
procesa $T_o(t)$ en valor  deseado o punto de control , en
presencia de variaciones en el flujo del fluido que se procesa ,
del cual la variación que presenta este proceso es la temperatura
de entrada o temperatura ambiente que infringe al sistema cuando
la puerta de alimentación del horno es aperturada.\\


\subsection{Medici\'on de temperatura}
La medida de temperatura consiste en una de las mediciones mas
comunes y mas importantes que se efectúan en los procesos
industriales.\\
Lresa a
una temperatura especificada la variación de resistencia en
ohmios del conductor por cada grado que cambia su temperatura.\\
La relación entre esos factores puede verse en la expresión lineal
siguiente.
\begin{equation*}
  R_t=R_O*(1+\alpha*t)
\end{equation*}
en la que:
\begin{description}
  \item[$R_O$] resistencia en ohmios a $0^{o}$C
  \item[$R_t$] resistencia en ohmios a $t^{o}$C
  \item[$\alpha$] coeficiente de temperatura de la resistencia
  cuyo valor entre $0^{o}$C y $100^{o}$C es de 0.003850 $\Omega$
  en la escala pr\'actica de temperaturas internacional (IPTS-68)
\end{description}
Si la relación resistencia -temperatura no es lineal la ecuación
general pasa a:
\begin{equation*}
  R_t=R_O*(1+A*t+B*t^{2})
\end{equation*}
ecuacion valida desde $0^{o}$C a $850^{o}$C.\\


\paragraph{Ventajas del PT100} Por otra parte los PT100 siendo levemente
mas costoso y mecánicamente no tan rígidos como las termocuplas
,las superan especialmente en aplicaciones de bajas temperaturas
(-100 a
$200^{o}$C ).\\
Los PT100 pueden fácilmente entregar precisiones de una décima de
grado con la ventaja que la PT100 no se descompone gradualmente
entregando lecturas erróneas , si no que normalmente se abre con
lo cual el dispositivo medidor detecta inmediatamente la falla de
sensor y da aviso.\\
Ademas las sondas PT100 pueden ser colocadas a cierta distancia
del punto de medici\'on sin mayor problema  ( hasta unos 30 metros
) utilizando cable de cobre convencional para hacer la extension.


\subsubsection{Circuitos de conexión del PT 100} Según el fabricante Arian
existen 3 modos de conexión para las Pt100, cada uno de ellos
requiere una adecuada configuraci\'on electr\'onica.

\paragraph{Conexión dos hilos}
Es el modo mas sencillo de conexi\'on (pero menos recomendado) es
con solo dos cables.\\En este caso las resistencias de los cables
$R_{c1}$ , $R_{c2}$ que unen el PT100 al instrumento de medici\'on
se suman generando un
error inevitable.\\
El lector medirá el total de $R(t)$+$R_{c1}$+$R_{c2}$ en vez de
$R(t)$

\paragraph{Conexión tres hilos} El modo de conexión de 3
hilos es el mas común y resuelve bastante bien el problema de
error generado por los cables.\\
El único requisito para esta conexi\'on es de que los tres cables
denominados cafe, azul ,verde tengan la misma resistencia
eléctrica pues el sistema de medici\'on se basa ( casi siempre )
en el puente de Wheatstone.
\paragraph{Conexión cuatro hilos}
El método de 4 hilos es el mas preciso de todos, los 4 cables
pueden ser distintos (distinta resistencia) pero el circuito
propuesto es mas costoso.
\begin{figure}[h!]
\begin{center}
 \includegraphics[width=10cm]{plani4.jpg}
\end{center}
\caption{Diagrama de conexión de cuatro hilos  ( Arian ) }
\end{figure}
Por los cables 1 y 4 se hace circular una corriente I conocida a
través de R(t)-PT100 provocando una diferencia de potencial V en
los extremos de R(t). Los cables 2 y 4 están conectados a la
entrada de un circuito de alta impedancia luego por estos cables
no circula corriente y por lo tanto la caída de potencial en los
cables Rc2 y Rc3 son cero y así dando la medici\'on  de R(t) en
voltios.

\subsubsection{Autocalentamiento PT100} Cualquiera que sea el
método de conexión , se debe hacer pasar un cierta corriente $I$
por el elemento sensor modo que de poder medir su resistencia.\\
E
\subsection{Controladores de tensi\'on alterna }
Un controlador de tension alterna es un convertidor que controla
lma de onda de la fuente antes de
alcanzar la carga.
\subsubsection{Funcionamiento básico}
En la figura siguiente se muestra un controlador de tension
e denomina antiparalelo o paralelo
inverso porque los SCR conducen corriente en sentidos opuestos. Un
triac es un equivalente a dos SCR en antiparalelo.\\
\begin{figure}[h!]
\begin{center}
 \includegraphics[width=10cm]{plani5.jpg}
\end{center}
\caption{a)Controlador de tension alterna  b)Formas de onda}
\end{figure}
Este circuito presenta algunas observaciones importantes que son:
\begin{itemize}
  \item Los SCR'S no pueden conducir simultáneamente
  \item La tension de carga es la misma que la tension de la
  fuente cuando esta activado cualquiera de los SCR .La tension de
  carga es nula cuándo esta desactivados los SCR.
  \item La tension del interruptor  $V_{SW}$ es nula cuando esta
  activado cualquiera de los SCR y es igual a la tension del
  generador cuando están desactivados los dos SCR.
  \item La corriente media en la fuente y en la carga es nula si
  se activan los dos SCR durante intervalos iguales de tiempo . La
  corriente media en cada SCR no es nula , debido a la corrriente
  unidireccional en los SCR.
  \item La corriente eficaz en cada  SCR es $1/raiz(2)$
  multiplicando por la corriente eficaz de carga si se activan los
  SCR durante intervalos iguales de tiempo.
\end{itemize}


\subsubsection{Triacs}
Este dispositivo tiene las siguientes características:
\begin{itemize}
  \item     La corriente y tensión eficaz de un Triac dependen del ángulo de conducción
  \item     A mayor ángulo de conducción, se obtiene a la salida mayor potencia
  \item      Los Triacs no son interruptores perfectos,
   necesitan un tiempo para pasar de corte a conducción y viceversa.
\end{itemize}

\subsection{Desarrollo del sistema de control}
El sistema propuesto se muestra en la figura donde el controlador
digital estará
basado en un PIC18F2550 que internamente tiene un ADC.\\
$R_{(kt)}$ es la se~nal que nos da el valor de temperatura que se
desea calentar en el horno,esta se~nal estará designada en un
valor digital dentro del microcontrolador.\\

\begin{figure}[h!]
\begin{center}
 \includegraphics[width=14cm]{plani7.jpg}
\end{center}
\caption{Sistema de control propuesto}
\end{figure}

Una vez evaluado el valor de correcci\'on $C_{(kt)}$ por el
controlador digital enviara este valor por un bus digital al
bloque denominado regulador de valor eficaz, el mismo que
entregara la energía a la
planta $G(s)$  denominada horno eléctrico.\\
La respuesta de la planta sera sensado por un transductor de
temperatura que este mandara la se~nal $T_{(t)}$ en forma
analógica al PIC para su realimentaci\'on mediante el conversor
análogo digital.


\section{Desarrollo de ecuaciones}
\subsection{Autovalores}
Sea la Matriz A : 
\begin{equation}
A=
\begin{pmatrix}
  a_{11}    & a_{12} \\
  a_{21}    & a_{22}       
\end{pmatrix}
\end{equation}
Aplicando su polinomio caracteristico 
$$|A -\lambda I_n|=0$$

\begin{equation}
\begin{vmatrix}
  a_{11}-\lambda    & + & a_{12}           \\
  a_{21}            & + & a_{22}-\lambda 
\end{vmatrix}
\end{equation}
Desarrollando

\begin{eqnarray}
  (a_{11}-\lambda)*(a_{22}-\lambda) - a_{12}*a_{21}=0   \\
  (a_{11})*(a_{22}-\lambda)-\lambda*(a_{22}-\lambda) - a_{12}*a_{21}=0  \\
  \lambda^2 +\lambda *(-a_{11}-a_{22})+(a_{11}*a_{22}-a_{12}*a_{21})
\end{eqnarray}
Definiendo las variables:
\begin{eqnarray}
b = (-a_{11}-a_{22})   \\
c = a_{11}*a_{22}-a_{12}*a_{21}  \\
\lambda^2 +\lambda *(b)+(c) = 0 
\end{eqnarray}
Del cual 
$\lambda =\lambda_1 ,\lambda_2 $
\subsection{Autovectoires}
sdf
\section{Aplicacion de las ecuaciones en Python}

dsd
\section{Resultados}
\section{CONCLUSIONES Y RECOMENDACIONES}



\begin{thebibliography}{1}
\bibitem{LaTeX}
Control Automatico de Procesos ( Teoria y Practica ),
\newblock {\em Carlos A. Smith , Versi\'on Espa~nola - Editorial LIMUSA }

\bibitem{LaTeX}
Instrumentacion Industrial
\newblock {\em Antonio Creus Sol\'e , Sexta Edici\'on }

\bibitem{LaTeX}
Electronica de Potencia
\newblock {\em Daniel W. Hart ,MADRID   2001  }

\bibitem{LaTeX}
PT100 su operaci\'on instalaci\'on y tablas
\newblock {\em http://www.arian.cl, Nota Tecnica 4  }


\end{thebibliography}


\end{document}
